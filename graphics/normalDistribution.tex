%%%%%%%%%%%%%%%%%%%%%%%%%%%%%%
% Template graphical model   %
% Author: Tom Lodewyckx    
% Modified by: William Murrah
%%%%%%%%%%%%%%%%%%%%%%%%%%%%%%
\documentclass[11pt]{report}
\usepackage{tikz}
\usetikzlibrary{fit,positioning}
% 0. Begin document
%%%%%%%%%%%%%%%%%%%
\begin{document}
\begin{figure}
\centering
\pgfmathdeclarefunction{gauss}{2}{%
  \pgfmathparse{1/(#2*sqrt(2*pi))*exp(-((x-#1)^2)/(2*#2^2))}%
}

\begin{tikzpicture}
\begin{axis}[
  no markers, domain=0:10, samples=100,
  axis lines*=left, xlabel=$x$, ylabel=$y$,
  every axis y label/.style={at=(current axis.above origin),anchor=south},
  every axis x label/.style={at=(current axis.right of origin),anchor=west},
  height=5cm, width=12cm,
  xtick={4,6.5}, ytick=\empty,
  enlargelimits=false, clip=false, axis on top,
  grid = major
  ]
  \addplot [fill=cyan!20, draw=none, domain=0:5.96] {gauss(6.5,1)} \closedcycle;
  \addplot [very thick,cyan!50!black] {gauss(4,1)};
  \addplot [very thick,cyan!50!black] {gauss(6.5,1)};


\draw [yshift=-0.6cm, latex-latex](axis cs:4,0) -- node [fill=white] {$1.96\sigma$} (axis cs:5.96,0);
\end{axis}
\end{tikzpicture}
\end{document}
