% Define a the counter cnt. Used to identify files generated for use
% with Gnuplot.
\newcounter{cnt}
\setcounter{cnt}{0}

% Macro for drawing one frame of the F-distribution animation.
\newcommand{\fdst}[4]{%
    % shade the critical region tail
    \draw[fill,orange]  (#1,0) -- plot[id=5\thecnt,domain=#1:5.5,samples=50]
        function {#4*(x**(0.5*#2-1))*((1+#2*x/#3)**(-0.5*#2-0.5*#3))}
            -- (5.5,0) -- cycle;

    % draw the F distribution curve
    \draw[color=blue!50!black,thick]
        plot[id=f4\thecnt,smooth,domain=0:5.5,samples=100]
        function {#4*(x**(0.5*#2-1))*((1+#2*x/#3)**(-0.5*#2-0.5*#3))};

    % draw the F axis
    \draw[->] (0,0) -- (6,0) node[right] {$F$};
    % label the critical region boundary
    \draw (#1,0) -- (#1,-0.02) node[below] {$#1$};
    % label 0
    \draw (0,0) -- (0,-0.02) node[below] {$0$};

    % add some lables for degrees of freedom and alpha level
    \draw (2,0.5) node[right] {$df_1 = #2$};
    \draw (2,0.4) node[right] {$df_2 = #3$};
    \draw (2,0.3) node[right] {$\alpha = 0.10$};

    % draw the y axis
    \draw[very thin,->] (0,0) -- (0,0.8);
}

\newcommand{\distpic}[3]{
    % First draw the upper distribution.
    % Shade the critical region:
    \fill[red!30] (0.658,0)  -- plot[id=f3,domain=0.658:3,samples=50]
        function {exp(-x*x*0.5/0.16)} -- (3,0) -- cycle;

    % Draw the normal distribution curve
    \draw[blue!50!black,smooth,thick] plot[id=f1,domain=-2:3,samples=50]
    function {exp(-x*x*0.5/0.16)};
    % Draw the x-axis
    \draw[->,black] (-2.2,0) -- (3.2,0);
    % Put some ticks and tick labels in:
    \foreach \x in {-2,-1,0,1,2,3}
    \draw (\x,0) -- (\x,-0.1) node[below] {$\x$};
    % Put in a label for the critical region boundary:
    \draw[red!50!black,thick] (0.658,0) node[below,yshift=-0.5cm] {0.658}
    -- (0.658,0.85);

    % Put in labels for accepting or rejecting the null hypothesis with
    % the corresponding regions:
    \draw[red!50!black,thick,->] (0.688,0.7) -- (1.3,0.7)
        node[anchor=south] {Reject  $H_0$};
    \draw[red!50!black,thick,->] (0.628,0.7) -- (-1,0.7)
        node[anchor=south]{\parbox{1.5cm}{\raggedright Fail to reject $H_0$}};

    % Add a label to the upper picture, when the null is true
    \draw (-3,1) node[above,draw,fill=green!30] {$H_0$ is true:};

    % Label the critical region with an alpha level:
    \draw[<-,thick] (0.75,0.05) -- (1.6,0.2) node[right,yshift=0.3cm]
    {\begin{tabular}{l} $\alpha=0.05$ \\ (Type I error rate) \end{tabular}};


    % Add a label showing the effect size between the two plots:
    \draw[very thin] (0,-1) -- (0,-0.5);
    \draw[<->,thick] (0,-1) node[left] {Effect size:  #1} -- (#1,-1);
    \draw[thick] (0,-.9) -- (0,-1.1);

    \draw[very thin] (#1,-1) -- (#1,-1.7);
    \draw[thick] (#1,-.9) -- (#1,-1.1);

    % Now draw the lower distribution showing the effect size:
    \begin{scope}[yshift=-3cm]
    % Shade the "reject H0" region red
    \fill[red!30] (0.658,0)  -- plot[id=f3\thecnt,domain=0.658:3,samples=50]
        function {exp(-(x-#1)*(x-#1)*0.5/0.16)} --
        (3,0) -- cycle;
        % Shade the "accept H0" region blue
    \fill[blue!30] (-2,0) -- plot[id=f4\thecnt,domain=-2:0.658,samples=50]
        function {exp(-(x-#1)*(x-#1)*0.5/0.16)} --
        (0.658,0) -- cycle;

        % Draw the shifted normal distribution:
    \draw[blue!50!black,smooth,thick] plot[id=f1\thecnt,domain=-2:3,samples=50]
            function {exp(-(x-#1)*(x-#1)*0.5/0.16)};

        % Draw the x-axis and put in some ticks and tick labels
    \draw[->,black] (-2.2,0) -- (3.2,0);
    \foreach \x in {-2,-1,0,1,2,3}
            \draw (\x,0) -- (\x,-0.1) node[below] {$\x$};

        % Draw and label the critical region boundary
    \draw[red!50!black,very thick] (0.658,0) node[below,yshift=-0.5cm] {0.658}
        -- (0.658,1.0);
    \draw[red!50!black,very thick,->] (0.688,0.7) -- (2.7,0.7)
        node[anchor=south west] {Reject  $H_0$};
    \draw[red!50!black,very thick,->] (0.628,0.7) -- (-0.5,0.7)
        node[anchor=south]{\parbox{1.5cm}{\raggedright Fail to reject $H_0$}};

    % Add a label to the lower picture, when the alternative hypothesis is true:
    \draw (-3,1) node[above,draw,fill=green!30] {$H_a$ is true:};

        % Add labels showing the statistical power and the Type II error rate:
    \draw[<-,thick] (1.5,0.1) -- (3,0.2) node[anchor=south west]
        {Power = \large #2};
    \draw[<-,thick] (0.4,0.1) -- (-1,0.2) node[left,yshift=0.3cm]
        {\begin{tabular}{l}
        $\beta$ = {\large #3} \\ (Type II error rate) \end{tabular}};
    \end{scope}
}
